\documentclass[thesis.tex]{subfiles} 
\begin{document}


\section{Before opinion polls}

The models and techniques examined in this paper rely primarily on polling data for an up-to-date estimate of popular opinion. But even without opinion polling, presidential elections can be predicted ahead of time using various indicators.

For example, \cite{Hibbs:2008aa} presents a Bread-and-Peace model of presidential elections, where he attempts to explain persistent fundamental determinants of election outcomes in a linear regression model. He looks at two variables, \begin{enumerate}
	\item the weighted-average growth of per-capita real personal disposable income over the previous term;
	\item and U.S. military fatalities in unprovoked, hostile deployments of American forces abroad,
\end{enumerate} and interpets elections as mostly referendums on the White House party's economic record, along with a substantial aversion to foreign wars. 

States popular votes can also be estimated ahead of time, using similar indicators such as economic factors, and further intuitive ones such as general political leaning or home-state advantage (\citealt{Campbell:1992aa,Campbell:2006aa}).

The models analyzed in this paper use these estimates as prior knowledge on election outcomes, to be incorporated with polling data.

\end{document}