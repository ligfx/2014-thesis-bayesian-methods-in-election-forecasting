\documentclass[thesis.tex]{subfiles}
\begin{document}

\section{Incorporating opinion polls}

In the American presidential elections, opinion polls are provided on both the national and state level by an assortment of different organizations. The models in this paper use polling data to estimate an accurate picture of true voter opinion.

Polls come with a few problems that models have to deal with. Polls themselves are not as accurate as them claim to be. People change their minds over the election cycle. Different organizations have different biases and inaccuracies, stemming from either political leanings or different methods of sampling. All organizations have to estimate who will end up voting to weight their polls appropriately.

We first examine the model described by \cite{Lock:2010aa}, which uses simple Bayesian inference to incorporate polls based on estimated accuracy some months before Election Day. It doesn't handle organizational biases, but it makes up for it with simplicity and the intuitiveness of the math.

After that, we examine the the model by \cite{Strauss:2007aa}, which uses Gibbs sampling to fit a more involved forecasting model, which estimates organizational biases on the fly.

\end{document}

\begin{comment}
- problem: how accurate/useful are they? how many months before campaign (problems 
  with changing mind, making up mind, uncertainty about who will vote)

  (Lock)'s approach: Bayesian inference

  (Strauss): Gibbs

- problem: organization biases. empirically, political leanings, or different 
  methods of polling/sampling, guessing who will actually vote

  (Strauss) Gibbs
  Nate Silver: historical data
\end{comment}