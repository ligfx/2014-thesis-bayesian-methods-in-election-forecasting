\documentclass[thesis.tex]{subfiles}
\begin{document}

\section{A Bayesian model accounting for variance in vote intention before the election}

In this section we examine the model proposed by \cite{Lock:2010aa}. The model empirically estimates variance in voter intention at different times before Election Day, and uses these estimates to incorporate polling data into forecasts derived from fundamentals \citep{Hibbs:2008aa}.

Let \(\alpha_t\) denote the true national proportion of people who intend to vote for the Democratic candidate \(t\) months before the election, and let \(\widehat{\alpha}_t\) denote an estimate of this value from a poll.

\bigskip

\noindent\textbf{Claim:} The poll variance is the variance of a mean of \(q\) Bernoulli samples, or of the ratio of a Binomial sample and \(q\). We see this because, for a Binomial(\(q\), \(\mu\)) sample, \(qy\), \begin{equation}\begin{aligned}
	\Var(y) = \frac{1}{q^2}qy(1-y) = \frac{y(1-y)}{q}.
\end{aligned}\end{equation}

\bigskip

\noindent\textbf{Claim (Law of total expectation):} \(\E (X) = \E(\E(X | Y))\), by\begin{equation}
\begin{aligned}
	\E(\E(X | Y)) &= \int \left[\int x p(x | y)\, dx\right] p(y)\, dy \\
	&= \iint x p(x, y)\, dx dy \\
	&= \int x \int p(x, y)\, dy dx \\
	&= \int x p(x) \, dx \\
	&= \E(X).
\end{aligned}
\end{equation}

\bigskip

\noindent\textbf{Claim (Law of total variance, or decomposition of variance):} \(\Var(Y) = \E(\Var(Y | X)) + \Var(\E(Y | X))\), by \begin{equation}
\begin{aligned}
	\Var(Y) &= \E(Y^2) - \E(Y)^2 \\
	&= \E(\E(Y^2 | X)) - \E(\E(Y | X))^2 \\
	&= \E\left(\Var(Y|X) + \E(Y | X)^2\right) - \E(\E(Y | X))^2 \\
	&= \E(\Var(Y | X)) + \Var(\E(Y | X)).
\end{aligned}
\end{equation}

So, by the laws of total variance and total expectation, we see that \begin{equation}\begin{aligned}
	\Var(\widehat{\alpha_t} | \alpha_0)
	&= \E(\Var(\widehat{\alpha_t} | \alpha_t) | \alpha_0) + \Var(\E(\widehat{\alpha_t} | \alpha_t) | \alpha_0) \\
	&= \E\left(\frac{\alpha_t(1- \alpha_t)}{n} | \alpha_0\right) + \Var(\alpha_t | \alpha_0) \\
	&= \frac{\E(\alpha_t | \alpha_0) - \E(\alpha_t^2 | \alpha_0)}{n} + Var(\alpha_t | \alpha_0) \\
	&= \frac{\E(\alpha_t | \alpha_0) + \E(\alpha_t | \alpha_0)^2}{n} + \frac{n-1}{n}\Var(\alpha_t | \alpha_0) \\
	&\approx \frac{\alpha_0 (1 + \alpha_0)}{n} + \Var(\alpha_t | \alpha_0).
\end{aligned}
\end{equation}

We can estimate \(\Var(\alpha_t | \alpha_0)\) empirically using polling data and outcomes from past elections, by calculating the empirical variance of polls and subtracting \(\alpha_0 (1 + \alpha_0) / n\) . Let \(\widehat{\alpha}_{t, i}\) and \(n_{t, i}\) denote estimated vote and sample size for the \(i\)-th poll in month \(t\), and \(N_t\) the number of polls in month \(t\). Then \begin{equation}
\begin{aligned}
\widehat{\Var}(\alpha_t | \alpha_0) = \frac{1}{N_t} \sum_{i=1}^{N_t}\left[ \left( \widehat{\alpha}_{t, i} - \alpha_0\right)^2 - \frac{\alpha_0 (1 - \alpha_0)}{n_{t,i}} \right].
\end{aligned}
\end{equation}

We also want to know \(\Var(\widehat{d}_{s, t} | d_{s, 0}),\) the variance in the relative position \(t\) months before the election of state \(s\). By a similar process, we estimate this using data from the last few election cycles, \begin{equation}\begin{aligned}
	\widehat{\Var}(d_{s, t} | d_{s, 0}) = \frac{1}{elections \cdot 50} \sum_{y}^{elections} \sum_{s=1}^{50} \left[ \left( \widehat{d}_{s, y, t} - d_{s, y, 0}\right)^2 - \frac{\alpha_{s, y, 0} (1 - \alpha_{s, y, 0})}{n_{s, y, t}} \right].
\end{aligned}\end{equation} Polls do not give us \(\widehat{d}_{s, y, t}\), so we  estimate it as \(\widehat{\alpha}_{s, y, t} - \widehat{\alpha}_{y, t}\). 

Now we have our estimates on variance, dependent only on time \(t\).

\subsection{States}

From before, we have our poll data \begin{equation}\begin{aligned}
\widehat{d_{s, t}} | d_{s, 0} \sim \Normal\left(d_{s, 0}, \frac{\alpha_{s, 0} (1 - \alpha_{s, 0})}{n_{s, t}} + \Var(d_{s, t} | d_{s, 0}\right),
\end{aligned}\end{equation} where we justify normality by the size of the polls. We also have a prior on state deviance from the national outcome, from our multilevel model, \begin{equation}\begin{aligned}
d_{s, 0} | d_{s, previous} \sim \Normal(d_{s, previous} | \Var(d_{s, 0} | d_{s, previous})).
\end{aligned}\end{equation} Because these are both Normal distributions, we end up with a conjugate posterior for state deviation \(d_{s, 0} | data\).

\subsection{National}

We have a distribution for poll data, \begin{equation}\begin{aligned}
\widehat{\alpha}_t | \alpha_0 \sim \Normal\left( \alpha_0, \frac{\alpha_0 (1 - \alpha_0)}{n_t} + \Var(\alpha_t | \alpha_0) \right),
\end{aligned}\end{equation} and some prior estimated from fundamentals \begin{equation}\begin{aligned}
	\alpha_0 \sim \Normal(\mu_0, \sigma_0^2),
\end{aligned}\end{equation} so again we end up with a simple conjugate posterior.

\begin{comment}
	TODO Should be weighted average on the Var(d_st, d_s0) stuff.
\end{comment}

\subsection{Monte Carlo}

\cite{Lock:2010aa} complete the model by estimating the distribution of Electoral College outcomes. They simulate 100,000 elections by \begin{enumerate}
	\item Randomly drawing a national popular vote from the national posterior distribution;
	\item Randomly drawing a deviation for each state, where \(\widehat{d}_{s, t} = \widehat{\alpha}_{s, 0} - \alpha_0\);
	\item And calculating the winner of the Electoral College.
\end{enumerate}

\begin{comment}

So, basically, first we look at what \cite{Lock:2010aa} are doing, because it's a nice example for Bayesian inference, and because the approach taken by \cite{Strauss:2007aa} uses Gibbs sampling and gets complicated.

Suppose a poll is centered on the result of an election \(\alpha,\) with some amount of variance \(sigma^2\). Given a prior on the result of the election, which is centered at \(\mu\) with some variance \(v^2\), Bayes' rule tells us that \[
	\alpha | y \sim \N \left[ \left(\frac{\mu}{v^2} + \frac{y}{\sigma^2}\right) \left( \frac{1}{v^2} + \frac{1}{\sigma^2} \right), \left( \frac{1}{v^2} + \frac{1}{\sigma^2} \right) \right].
\]

How do we find out that variance term? Or, how much information does a poll have? (the variance from the election result)

Remember our problems: low reliability earlier in a campaign year, organizational biases.

First we'll talk about low reliability earlier in a campaign year

\end{comment}

\end{document}