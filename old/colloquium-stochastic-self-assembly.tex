\documentclass[12pt]{article}

\usepackage[margin=1.5in]{geometry}

\usepackage{titling}
\setlength{\droptitle}{-50pt}

\title{\textbf{Summary: Stochastic self-assembly and cluster distributions in biology}}
\author{Michael Maltese}
\date{14 December 2013}

\usepackage{parskip}

\begin{document}

\maketitle

\emph{I attended this talk on 4 December 2013 at Davidson Lecture Hall. Tom Chou from UCLA presented.}

One open problem in biology is modeling self-assembly, where objects combine into groups, such as when peptides on membranes self-assemble into pores through the membrane. The general approach follows the mass-action analysis of Becker and D\"oring, but there has been very little stochastic treatment of the problem. Chou compares his stochastic approach and the Becker-D\"oring approach under two scenarios: that of heterogeneous nucleation, where self-assembly is catalyzed by impurities; and homogeneous nucleation, where self-assembly is spontaneous. The stochastic approach involved developing trees of possible states of groups and monomers, transitioning according to specified probabilities, then running ``Kinetic Monte-Carlo simulations'' to find the full distribution. Chou finds that in the case of heterogeneous nucleation, the error between his KMC approach and the B-D approach is not high. In the case of homogeneous nucleation, however, the KMC approach does far better than B-D, most importantly showing the dynamics of emulsification, where adding a single extra monomer to a solution can significantly change the equilibrium distribution of groups and sizes of groups, which the B-D approach does not see.

\end{document}